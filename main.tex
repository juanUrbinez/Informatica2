\documentclass{article}
\usepackage[utf8]{inputenc}
\usepackage[spanish]{babel}
\usepackage{listings}
\usepackage{graphicx}
\graphicspath{ {images/} }
\usepackage{cite}

\begin{document}

\begin{titlepage}
    \begin{center}
        \vspace*{1cm}
            
        \Huge
        \textbf{Nociones de la memoria del computador}
            
        \vspace{0.5cm}
        \LARGE
        Subtítulo
            
        \vspace{1.5cm}
            
        \textbf{Juan Andres Urbiñez Gómez}
            
        \vfill
            
        \vspace{0.8cm}
            
        \Large
        Despartamento de Ingeniería Electrónica y Telecomunicaciones\\
        Universidad de Antioquia\\
        Medellín\\
        Septiembre de 2020
            
    \end{center}
\end{titlepage}

\tableofcontents
\newpage
\section{Definicion de memoria del computador}
La memoria del computador se podria definir como dispositivo donde se almacena informacion temporalmente para luego ser llevada a procesador, estos usualmente su capacidad y velocidad son opuestamente procorcionales, es decir, a mayor capacidad, menor velocidad y viceversa.
\section{Tipos de memoria conocidas anterior a investigacion} 
Antes de hacer esta investigacion solo conocia muy pocos tipos de memoria siendo, RAM, cache, VRAm

\subsection{Memoria RAM}
Random Acess Memory o por sus iniciales <<RAM>> en pocas palabras es la que toma parte de informacion del disco duro de algun proceso activo en el computador, y almacenarlo temporalmente para luego procesarlo. Al ser memoria temporal, al apagar el computador la informacion en la RAM  desaparece

\subsection{Memoria cache}
La memoria cache a diferencia de la <<RAM>> esta muy cerca de la CPU para la informacion requerida por la CPU de manera inmediata esta 


\section{Gestion de memoria en el computador}

\section{Rapidez de memoria ¿Que la causa?}

\section{Conclusión} \label{conclulsion}

\bibliographystyle{IEEEtran}
\bibliography{references}

\end{document}

